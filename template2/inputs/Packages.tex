%-------------------------------------------------------------
% Packages
%-------------------------------------------------------------
\usepackage[
	automark,								 	% Kapitel in Kopfzeile
	headsepline,								% Trennlinie unter Kopfzeile
	ilines										% Trennlinie linksbündig ausrichten
]{scrpage2}

\usepackage[ngerman]{babel}						% Sprachanpassungen

\usepackage{makeidx}
\usepackage[utf8]{inputenc}					% erweiterter Eingabezeichensatz
\usepackage[T1]{fontenc}						% erweiterter T1 Zeichenvorrat
\usepackage{textcomp} 							% Euro-Zeichen etc.
%\usepackage[textstyle]{SIunits}					% \micro für µ oder \degree für °
\usepackage{unitsdef}							% \micro für µ oder \degree für ° und weitere selbst definierte Einheiten
\usepackage{wasysym}							% + Unterstützung von Einheiten im Text
\usepackage{lmodern}							% bessere Fonts (manuell installieren cm-super)
\usepackage{relsize}							% Schriftgröße relativ festlegen

%\usepackage{cmbright}							% Schriftart CM bright

\usepackage[dvips,final]{graphicx}				% Einbinden von jpg ermöglichen
\graphicspath{{Grafik/}}						% hier liegen die Bilder des Dokuments
\usepackage{caption}
\usepackage{subcaption}
\usepackage{sidecap}							% seitliche captions
\usepackage{pstool}								% Einbinden von Matlab figures + .tex (matlabfrag)
\usepackage[acronym,toc,automake]{glossaries}			% Glossar, Symbolverzeichnis


\usepackage{setspace}							% einfache Festlegung des Zeilenabstands: \onehalfspacing, \doublespacing, \singlespacing
\usepackage{geometry}							% Seitenränder festlegen
\usepackage{chngcntr}							% fortlaufendes Durchnummerieren der Fußnoten

\usepackage{multirow}

\usepackage{paralist}							% kompaktere Listen, \begin{compactitem} anstatt \beginn{itemize}
\usepackage{longtable}							% Für Tabellen (ähnlich tabular) allerdings auch mehrseitige Tabellen möglich
\usepackage{array}								% weitere Möglichkeiten in Tabelle


\usepackage{xspace}								% Leerzeichen hinter parameterlosen Makros werden nicht als Makroendzeichen interpretiert



\usepackage[									% Zitierstil
	numbers,
	nonamebreak
]{natbib}

\usepackage[									% PDF-Verlinkungen
    bookmarks,
    bookmarksopen=true,
    colorlinks=true,							% diese Farbdefinitionen zeichnen Links im PDF farblich aus
    linkcolor=black, 							% einfache interne Verknüpfungen
    anchorcolor=blue,							% Ankertext
    citecolor=blue, 							% Verweise auf Literaturverzeichniseinträge im Text
    filecolor=magenta, 							% Verknüpfungen, die lokale Dateien öffnen
    menucolor=red, 								% Acrobat-Menüpunkte
    urlcolor=blue, 
												%% diese Farbdefinitionen sollten für den Druck verwendet werden (alles schwarz):
%    linkcolor=black,							% einfache interne Verknüpfungen
%    anchorcolor=black, 						% Ankertext
%    citecolor=black, 							% Verweise auf Literaturverzeichniseinträge im Text
%    filecolor=black, 							% Verknüpfungen, die lokale Dateien öffnen
%    menucolor=black, 							% Acrobat-Menüpunkte
%    urlcolor=black, 
    backref,
    plainpages=false, 							% zur korrekten Erstellung der Bookmarks
    pdfpagelabels, 								% zur korrekten Erstellung der Bookmarks
    hypertexnames=false,
    breaklinks=true,					% zur korrekten Erstellung der Bookmarks
    %linktocpage 								% Seitenzahlen anstatt Text im Inhaltsverzeichnis verlinken
]{hyperref}


%\hypersetup{
%    pdftitle={\titel},
%    pdfauthor={\autor},
%    pdfcreator={\autor},
%    pdfsubject={\titel},
%    pdfkeywords={\titel},
%}

\usepackage{blindtext}							% Blindtext Deutsch
\usepackage{lipsum}								% Blindtext "lorem ipsum"
\usepackage{pdfpages}

%-------------------------------------------------------------
% Backup
%-------------------------------------------------------------


% Befehle aus AMSTeX für mathematische Symbole z.B. \boldsymbol \mathbb
%-------------------------------------------------------------
\usepackage{amsmath,amsfonts}
\usepackage{pifont,mathptmx,charter,courier}
%\usepackage[scaled]{helvet}


% Weitere
%-------------------------------------------------------------
%\usepackage{ifthen}							% bei der Definition eigener Befehle benötigt
%\usepackage{todonotes}							% definiert u.a. Befehle \todo und \listoftodos

%\usepackage{makeidx}							% Index / Einträge im Quelltext markieren durch \index
												% An der Stelle an der Der Index erscheinen soll \printindex
%\makeindex										% texmaker mac, Benutzer Befehle "/usr/texbin/makeindex" %

%\usepackage{multicol}							% Erweiterte Funktionalität für Spalten
%\usepackage{multirow}							% In Tabelle: Zelle über mehrere Zeilen
%\usepackage{multicolumn}						% In Tabelle: Zelle über mehrere Spalten
%\usepackage{colortbl}							% Farbige Tabellen
%\usepackage{rotating}							% gedrehte Tabelle

%\usepackage{floatflt}							% umfließen von Bildern

\usepackage{url}								% URL verlinken, lange umbrechen (das macht auch hyperref)

%\usepackage{bibgerm}							% Zitierstil
%\usepackage[numbers,square]{natbib}			% Zitierstil
%\usepackage{natbib}							% Zitierstil
%\usepackage{cite}								% Zitierstil
%\newcommand{\shortcite}[1]{\cite{#1}}			% Zitierstil

%\usepackage{ragged2e}
%\newcolumntype{w}[1]{>{\raggedleft\hspace{0pt}}p{#1}}
%\usepackage{lscape}

%\usepackage{color}
\usepackage{xcolor}							% Farben definieren
\definecolor{hellgelb}{rgb}{1,1,0.9}
\definecolor{hellblau}{rgb}{1,0.8,0}
\definecolor{hellgrau}{rgb}{0.8,0.8,0.8}
\definecolor{colKeys}{rgb}{0,0,1}
\definecolor{colIdentifier}{rgb}{0,0,0}
\definecolor{colComments}{RGB}{0,139,0}
\definecolor{colString}{rgb}{0,0.5,0}

\usepackage{listings}							% Einbinden von Programmcode
\lstset{
    float=hbp,
    basicstyle=\scriptsize\bfseries\ttfamily\color{black},%\small\smaller,
    identifierstyle=\color{colIdentifier},
    keywordstyle=\color{blue},
    stringstyle=\color{colString},
    commentstyle=\color{gray},
    columns=flexible,
    tabsize=2,
    frame=single,
    extendedchars=true,
    showspaces=false,
    showstringspaces=false,
    numbers=left,
    numberstyle=\tiny,
    breaklines=true,
    backgroundcolor=\color{hellgelb},
    breakautoindent=true,
    caption=Testprogramm,
    language=python
}