% Erstellung eines Index und Abkürzungsverzeichnisses aktivieren
%-------------------------------------------------------------


\makeindex %														%Ein eigenes Symbolverzeichnis erstellen

%\renewcommand*{\glstextformat}[1]{\textit{#1}}
%\renewcommand*{\firstacronymfont}[1]{\normalfont{#1}}
%\renewcommand*{\acronymfont}[1]{\normalfont{#1}}
%\defglsdisplay{\emph{#1#4}}
%\defglsdisplayfirst{\emph}

%\renewcommand*{\firstacronymfont}[1]{#1}
%\renewcommand*{\acronymfont}[1]{\emph{#1}}


\newglossary[slg]{symbolslist}{syi}{syg}{Symbolverzeichnis}		%Style Symbolverzeichnis definieren
\newglossarystyle{symbver}{										% put the glossary in a longtable environment:
\renewenvironment{theglossary}
{\begin{longtable}{p{.10\textwidth}p{.7\textwidth}p{.1\textwidth}}}
{\end{longtable}}
\renewcommand*{\glossaryheader}{{Symbol}& {Beschreibung} & {Einheit}\\ \hline}

\renewcommand*{\glossaryentryfield}[4]{\glstarget{##1}\\[0.1cm]{##2}&{##3}&{##4}}
																% \\[0.2cm] Zeilenabstand zwischen Einträgen
\renewcommand*{\glossarysubentryfield}[6]{\glossaryentryfield{##2}{##3}{##4}{##5}}
\renewcommand*{\glsgroupskip}{}}

\renewcommand*{\glspostdescription}{}							% Den Punkt am Ende jeder Beschreibung deaktivieren

%\renewcommand*{\glstextformat}[1]{\textit{#1}}
\renewcommand*{\firstacronymfont}[1]{\normalfont{#1}}
%\renewcommand*{\acronymfont}[1]{\normalfont{#1}}
%\defglsdisplay{\emph{#1#4}}
\defglsdisplayfirst{\emph}

\newglossaryentry{greekletter}{name={\textbf{\rlap{griechisches Alphabet}}},description={}}
																% trennt Symbolverzeichnis in latein und griechisch

\newglossaryentry{romanletter}{name={\textbf{\rlap{lateinisches Alphabet}}},description={}}
																% trennt Symbolverzeichnis in latein und griechisch

\makeglossaries

		

% Einrichtung unter TeXnicCenter!!!
% Als Postprozess (Ausgabe, Ausgabeprofile definieren, Nachbearbeiten Einstellen:
% Anwendung: Programm\MiKTeX\bin\makeindex.exe
% Argumente:	für	Abkürzungsverzeichnis:		-s "%tm.ist" -t "%tm.alg" -o "%tm.acr" "%tm.acn"
%					Glossar:					-s "%tm.ist" -t "%tm.glg" -o "%tm.gls" "%tm.glo"
%					Symbolverzeichnis:			-s "%tm.ist" -t "%tm.slg" -o "%tm.syi" "%tm.syg"

% Einrichtung unter TexMaker Mac
% Benutzer > Benutzer Befehle > "/usr/texbin/makeglossaries" %

